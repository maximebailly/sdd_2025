\documentclass[sans,14pt]{beamer}
\usepackage{pgfpages}
%\setbeameroption{show notes on second screen}
%\setbeameroption{show notes}
\setbeameroption{hide notes}

\usepackage{etex}

\mode<presentation>
{
  \usetheme{Copenhagen}
  \usecolortheme{beaver}
  \usefonttheme[onlymath]{serif}
}

%% Change bullet points style
\setbeamertemplate{itemize items}{{\textemdash}}
\setbeamercolor*{item}{fg=MyDarkGrey}

%% Remove navigation symbols
\setbeamertemplate{navigation symbols}{}
\setbeamertemplate{headline}{}

%% Use only frame number (no total frame count) in footer
\setbeamertemplate{footline}{%
  \raisebox{5pt}{\makebox[\paperwidth]{\hfill\makebox[20pt]{
        \scriptsize\insertframenumber}}}
}

% Adjust margin of itemize environment
\setlength{\leftmargini}{0pt}
\setlength{\leftmarginii}{10pt}
\setlength{\leftmarginiii}{17pt}

%% Adjust font sizes
\setbeamerfont{frametitle}{size=\Large}
\setbeamerfont{normal text}{size=\small}
\setbeamerfont{itemize/enumerate body}{size=\small}
\setbeamerfont{itemize/enumerate subbody}{size=\small}
\setbeamerfont{itemize/enumerate subsubbody}{size=\footnotesize}

%%% To use the Cabin-Condensed fonts
%%% More fonts: see http://www.tug.dk/FontCatalogu
% \usepackage[latin1]{inputenc}
% \usepackage[T1]{fontenc}
% \usepackage[sfdefault]{cabin} 
\usepackage{PTSans} 
\usepackage{PTSansNarrow} 
\renewcommand*\familydefault{\sfdefault} %% Only if the base font of the document is to be sans serif
\usepackage[T1]{fontenc}

\newenvironment{ptnormal}{\fontfamily{PTSans-TLF}\selectfont}{\par}


\usepackage{amsmath}
\usepackage{amsfonts}
\usepackage{amssymb}
\usepackage{amsthm}
\usepackage{bm}
\usepackage{booktabs}
\usepackage{color}
\usepackage{epstopdf}
\usepackage{fancyhdr}
\usepackage{fancybox}
\usepackage{graphicx}
\usepackage{multirow}
\usepackage[normalem]{ulem}
\usepackage{url}
%


% tikzmark command, for shading over items
\usepackage{tikz}
\usetikzlibrary{calc}
\newcommand{\tikzmark}[1]{\tikz[overlay,remember picture] \node (#1) {};}
\usetikzlibrary{decorations.pathreplacing} % for curly braces
\usepackage[framemethod=tikz]{mdframed} % for fancy title page and other boxes

%% Subfigures
\usepackage[labelformat=simple,caption=false,textfont=scriptsize]{subfig}
\renewcommand{\thesubfigure}{\relax} % do not number captions of
                                     % subfloats



%%% Colors
%\usepackage{color}
\definecolor{MyBlue}{rgb}{0.27,0.62,0.73}%{0.,0.2,0.4} 
\definecolor{Aubergine}{rgb}{0.47,0.13,0.44} % 119 33 111
\definecolor{MyTurq}{rgb}{0.,0.8,0.8} 
\definecolor{MyDarkGrey}{rgb}{0.3,0.3,0.3} 
\definecolor{MyMedGrey}{rgb}{0.6,0.6,0.6} 
\definecolor{MyLightGrey}{rgb}{0.95,0.95,0.95} 
\definecolor{MyDarkRed}{rgb}{0.8,0.,0.} 
\definecolor{MyOrange}{rgb}{0.93,0.56,0.25}
\definecolor{MyPink}{rgb}{0.73,0.27,0.39}
\definecolor{MyLightYellow}{HTML}{FFFFCC}
%\usepackage[colorlinks=true,citecolor=MyTurq]{hyperref}
% \AtEveryCite{\color{MyMedGrey}}

\newcommand{\blue}[1]{{\color{MyBlue}{\textbf{#1}}}}
\newcommand{\red}[1]{{\color{MyOrange}{\textbf{#1}}}}
\newcommand{\black}[1]{{\color{MyDarkGrey}{\textbf{#1}}}}
\newcommand{\yellow}[1]{{\color{MyLightYellow}{\textbf{#1}}}}

\newcommand{\TODO}{{\color{MyDarkRed}{TODO~}}}

\setbeamercolor{normal text}{fg=MyDarkGrey}
\setbeamercolor{frametitle}{fg=Aubergine}

%%% ---- Titre des slides  (underline) ----------------
\setbeamertemplate{frametitle}{%
    \usebeamerfont{frametitle}\insertframetitle\strut%
    \vskip-.05\baselineskip%
    \leaders\vrule width \paperwidth\vskip1.pt%
    \vskip0pt%
    \nointerlineskip
}
%%% ---------------------------------------------------


% --- Math symbols -------------------------------------------------------------
\newcommand{\EE}{{\mathbb E}}
\newcommand{\PP}{{\mathbb P}}
\newcommand{\NN}{{\mathbb N}}
\newcommand{\RR}{{\mathbb R}}

\newcommand{\avec}{\vec{a}}
\newcommand{\alphavec}{\vec{\alpha}}
\newcommand{\betavec}{\vec{\beta}}
\newcommand{\cvec}{\vec{c}}
\newcommand{\fvec}{\vec{f}}
\newcommand{\muvec}{\vec{\mu}}
\newcommand{\rvec}{\vec{r}}
\newcommand{\thetavec}{\vec{\theta}}
\newcommand{\xvec}{\vec{x}}
\newcommand{\yvec}{\vec{y}}
\newcommand{\wvec}{\vec{w}}
\newcommand{\zvec}{\vec{z}}

% \newcommand{\amat}{{\mathbf A}}}
% \newcommand{\dmat}{{\mathbf D}}
% \newcommand{\gmat}{{\mathbf G}}
% \newcommand{\kmat}{{\mathbf K}}
% \newcommand{\lmat}{{\mathbf L}}
% \newcommand{\mmat}{{\mathbf M}}
% \newcommand{\wmat}{{\mathbf W}}
% \newcommand{\xmat}{{\mathbf X}}

\newcommand{\dset}{\mathcal{D}}
\newcommand{\fcal}{\mathcal{F}}
\newcommand{\ncal}{\mathcal{N}}
\newcommand{\rcal}{\mathcal{R}}
\newcommand{\sset}{\mathcal{S}}
\newcommand{\vset}{\mathcal{V}}
\newcommand{\tset}{\mathcal{T}}
\newcommand{\xcal}{\mathcal{X}}
\newcommand{\ycal}{\mathcal{Y}}

\newcommand{\lzeronorm}[1]{\left|\left|#1\right|\right|_0}
\newcommand{\lonenorm}[1]{\left|\left|#1\right|\right|_1}
\newcommand{\ltwonorm}[1]{\left|\left|#1\right|\right|_2}

\DeclareMathOperator*{\argmin}{arg\,min}
\DeclareMathOperator*{\argmax}{arg\,max}
% ------------------------------------------------------------------------------







\newcommand{\mytitle}{ECUE21.2 Science des données (DATA)}
\author{Chlo\'e-Agathe~Azencott et Emilie Chautru}
\date{\small Juin 2024}

\begin{document}
{
  \begin{frame}[plain]
    \fontfamily{PTSansNarrow-TLF}\selectfont
    % \begin{center}
    %   {\ptnormal{\textit{ECUE21.2 Science des données}}}
    % \end{center}

    %\vspace{100pt}

      {\Large \bf \mytitle}

      \insertauthor

      % {\footnotesize
      %   \centerline{Center for Computational Biology (CBIO)}
      
      %   \centerline{Mines Paris PSL -- Institut Curie -- INSERM U900}

      %   \centerline{PSL Research University \& PR[AI]RIE, Paris, France}
      % }

      % \centerline {\insertdate}


      % \vspace{-15pt}

      % \begin{center}
      %   \begin{tabular}[b]{ccc}
      %     {\scriptsize \href{http://cazencott.info}{http://cazencott.info}} &
      %     {\scriptsize \href{chloe-agathe.azencott@minesparis.psl.eu}
      %       {chloe-agathe.azencott@minesparis.psl.eu}} & 
      %     {\scriptsize \href{https://lipn.info/@cazencott}{@cazencott@lipn.info}} \\
      %   \end{tabular}
      % \end{center}
\end{frame}

% Do not count title slide when numbering frames
\addtocounter{framenumber}{-1}

\begin{frame}
  \frametitle{Objectifs}
  \begin{itemize}
  \item \blue{Démystifier} science des données, big data, intelligence artificielle
  \item Poser les bases \blue{mathématiques} et \blue{algorithmiques} de
    \begin{itemize}
      \setlength{\itemsep}{30pt}      
    \item la \red{statistique inférentielle} 
    \item l'\red{apprentissage automatique}
    \end{itemize}
    
  \end{itemize}
\end{frame}

\begin{frame}
  \frametitle{Moyens}
  \begin{itemize}
  \item \black{Modalités}
    \begin{itemize}
      \setlength{\itemsep}{30pt}
    \item 8 \blue{amphis}
    \item 6 \blue{petites classes} \red{à préparer}
    \item 2 \blue{séances de projet numérique}
    \end{itemize}
  \end{itemize}
\end{frame}

\begin{frame}
  \frametitle{Moyens}
  \begin{itemize}
\item \black{Ressources}
    \begin{itemize}
    \item \blue{Github:} \href{https://github.com/chagaz/sdd\_2025}{https://github.com/chagaz/sdd\_2025}
    \item[] Poly, sujets de petites classes et de projet et leurs sources
    \end{itemize}
    \vspace{30pt}
  \begin{itemize}
    \item \blue{Moodle:} \href{https://moodle.psl.eu/course/view.php?id=30665}{https://moodle.psl.eu/course/view.php?id=30665}
    \item[] Ressources complémentaires, dépôt de projet
    \end{itemize}
  \end{itemize}
\end{frame}

\begin{frame}
  \frametitle{Évaluation}
  \begin{itemize}
  \item \blue{Examen sur table} 10 juillet (70\%)
    \begin{itemize}
    \item Tous documents papier autorisés
    \end{itemize}
  \pause
  \item \blue{Projet numérique} 10 juillet (30\%)
    \begin{itemize}
    \item En binôme
    \item Deux séances dédiées (20 juin, 7 juillet) \red{présence notée}
    \item Utilisation d'agents conversationnels/assistants de code \red{prohibée} et \red{sanctionnée}
      \begin{itemize}
      \item Entrave à l'apprentissage 
      \item Possibles effets négatifs sur les capacités cognitives
       et la qualité du code
      \item Enjeux éthiques, écologiques, géopolitiques (voir Chap. 6)
      \end{itemize}
    \item<3-> Alternatives : PCs, documentation, recherche internet standard, stackoverflow, équipe enseignante, camarades.
    \end{itemize}
  \end{itemize}
  \only<2->{\vskip 0pt plus 1filll \scriptsize
  Bastani et al. (2024), Generative AI Can Harm Learning. The Wharton School Research Paper\\
  Zhai et al. (2024), The effects of over-reliance on AI dialogue systems on students' cognitive abilities. SLE\\
Perry et al. (2023), Do Users Write More Insecure Code with AI Assistants? CCS'23}
\end{frame}

\end{document}

%%% Local Variables:
%%% mode: latex
%%% TeX-master: t
%%% End:
